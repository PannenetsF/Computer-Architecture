\documentclass[lang=cn,11pt,a4paper,cite=authoryear]{elegantpaper}

% 微分号
\newcommand{\dd}[1]{\mathrm{d}#1}
\newcommand{\pp}[1]{\partial{}#1}

\newcommand{\homep}[1]{\Large\textbf{Problem #1}}
\newcommand{\subhome}[1]{\large\textbf{SubProblem #1}}

% FT LT ZT
\newcommand{\ft}[1]{\mathscr{F}[#1]}
\newcommand{\fta}{\xrightarrow{\mathscr{F}}}
\newcommand{\lt}[1]{\mathscr{L}[#1]}
\newcommand{\lta}{\xrightarrow{\mathscr{L}}}
\newcommand{\zt}[1]{\mathscr{Z}[#1]}
\newcommand{\zta}{\xrightarrow{\mathscr{Z}}}

% 积分求和号

\newcommand{\dsum}{\displaystyle\sum}
\newcommand{\aint}{\int_{-\infty}^{+\infty}}

% 简易图片插入
\newcommand{\qfig}[3][nolabel]{
  \begin{figure}[!htb]
      \centering
      \includegraphics[width=0.6\textwidth]{#2}
      \caption{#3}
      \label{#1}
  \end{figure}
}

% 表格
\renewcommand\arraystretch{1.5}


% 日期

\lstset{
  mathescape=false
}

\title{计算机体系架构\quad Lab03}
\author{范云潜 18373486}
\institute{微电子学院 184111 班}
\date{\zhtoday}

\begin{document}

\maketitle

\homep{ex1}

分为两个状态机, CE 控制保持与计数的状态,若是处在计数的状态,计数的循环(第二个状态机)进行工作,如 \figref{01} 。

\qfig[01]{l3p1.png}{状态机}

\homep{ex2}

波形图如 \figref{02} ,其中加法延时为 \(2/10 clk\) ,寄存器延时为 \(1/10 clk\) 。

\qfig[02]{l3p2.png}{状态机波形图}

这是一个简单的 \lstinline{fib} 计算电路。

Reg2 在 \lstinline{rst} 后的第 i 个周期,保存 \lstinline{fib(i)} 的值。

使用 C 进行计算。

\begin{lstlisting}
    int fib(clk) {
        int r0 = 0;
        int r1 = 1;
        int ret;
        if (clk <= 1) 
            ret = clk;
        for (int i = 2; i < clk; ++i) {
            ret = r0 + r1;
            r0 = r1;
            r2 = ret;
        }
        return ret;
    }
\end{lstlisting}

使用 Verilog 进行建模,硬件部分:

\begin{lstlisting}
`timescale 10ns/10ns

module c (
    clk,
    rst
);

input clk;
input rst;

reg [15:0] r1;
reg [15:0] r2;

wire [15:0] si, si1, si2;
assign #2 si = si1 + si2;
assign #1 si1 = r1;
assign #1 si2 = r2;

// assign 

always @(posedge clk or rst) begin
    if (rst) begin
        r1 <= 1;
        r2 <= 0;    
    end
    else begin
        r1 <= si;
        r2 <= si1;
    end
end
    
endmodule

\end{lstlisting}

激励部分:

\begin{lstlisting}

`include "test.v"
module moduleName (

);

reg clk;
reg rst;

always #10 clk = ~clk;


c u(clk, rst);

initial begin
    $dumpfile("test.vcd");
    $dumpvars;
    clk = 0;
    rst = 1;
    #40 rst = 0;
    #30;
    #1000;
    $finish;
end

endmodule

\end{lstlisting}

% Start Here

% End Here

\end{document}