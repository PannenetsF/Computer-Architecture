\documentclass[lang=cn,11pt,a4paper,cite=authoryear]{elegantpaper}

% 微分号
\newcommand{\dd}[1]{\mathrm{d}#1}
\newcommand{\pp}[1]{\partial{}#1}

\newcommand{\homep}[1]{\Large\textbf{Problem #1}}
\newcommand{\subhome}[1]{\large\textbf{SubProblem #1}}

% FT LT ZT
\newcommand{\ft}[1]{\mathscr{F}[#1]}
\newcommand{\fta}{\xrightarrow{\mathscr{F}}}
\newcommand{\lt}[1]{\mathscr{L}[#1]}
\newcommand{\lta}{\xrightarrow{\mathscr{L}}}
\newcommand{\zt}[1]{\mathscr{Z}[#1]}
\newcommand{\zta}{\xrightarrow{\mathscr{Z}}}

% 积分求和号

\newcommand{\dsum}{\displaystyle\sum}
\newcommand{\aint}{\int_{-\infty}^{+\infty}}

% 简易图片插入
\newcommand{\qfig}[3][nolabel]{
  \begin{figure}[!htb]
      \centering
      \includegraphics[width=0.6\textwidth]{#2}
      \caption{#3}
      \label{#1}
  \end{figure}
}

% 表格
\renewcommand\arraystretch{1.5}


% 日期

\lstset{
  mathescape=false
}

\title{计算机体系架构\quad 第八周作业}
\author{范云潜 18373486}
\institute{微电子学院 184111 班}
\date{\zhtoday}

\begin{document}

\maketitle

作业内容:7.1,7.3,7.6,7.11,7.25,7.32,7.36。

\homep{7.1}

如果不考虑成本的话,更大的 SRAM 会占据更大额度片上面积,进而影响其他的模块,并且缺少 RAM 接口,难以兼容以 RAM 接口为主的 DMA 模块,速度成为瓶颈。

\homep{7.3}

对数据来说,时间连续,但是空间性不强,哈希表是一个很好的例子。

\begin{lstlisting}
Get: HashMap map
data = map[key]
for op in {find_index, find_value, ...}:
    op.answer = op(data)
\end{lstlisting}

\homep{7.6} 

对指令来说,时间和空间性都不强,也就是指令的跳跃性很强。

\begin{lstlisting}
Get: UserInput keyboard
data, choice = read(keyboard)
case(choice):
    type_1:
        multiply(data[0], data[1])
    type_2:
        add(*((int*)data[2]), 0xff3e)
    type_3:
        copy(from=data, to=new_data)
\end{lstlisting}


\homep{7.11}    

在不进行优化的情况下,通过时钟函数进行计时,代码如下,y优化比例大概有 2.5 倍。

\begin{lstlisting}
#include <stdio.h>
#include <time.h>

int main()
{
    static int array[10000][100000];
    clock_t start, end;
    start = clock();
    double t1, t2;
    for (int i = 0; i < 10000; ++i)
    {
        for (int j = 0; j < 100000; ++j)
        {
            array[i][j] = 2 * array[i][j];
        }
    }
    end = clock();
    t1 = end - start;
    printf("the cost of for-row is %d\n", end - start);

    start = clock();
    for (int i = 0; i < 100000; ++i)
    {
        for (int j = 0; j < 10000; ++j)
        {
            array[j][i] = 2 * array[j][i];
        }
    }
    end = clock();

    t2 = end - start;
    printf("the cost of for-col is %d\n", end - start);

    printf("the ratio is %f", t1 / t2);

    return 0;
}

>>> the cost of for-row is 7779271
>>> the cost of for-col is 19284270
>>> the ratio is 0.403400                
\end{lstlisting}

\homep{7.25}

\begin{lstlisting}
addr:
    2:  empty and load to way1(1-8)
    3:  hit 
    11: empty and load to way2(9-16) 
    16: hit 
    21: miss and load to way1(17-24)
    13: hit
    64: miss and load to way1(57-64)
    48: miss and load to way2(41-48)
    19: miss and load to way1(17-24)
    11: miss and load to way2(9-16)
    3:  miss and load to way1(1-8)
    22: miss and load to way2(17-24)
    4:  hit 
    27: miss and load to way2(25-32)
    6:  hit 
    11: miss and load to way2(9-16)
buffer:  
    way1: 1   2   3   4   5   6   7   8
    way2: 9   10  11  12  13  14  15  16
\end{lstlisting}

\homep{7.32}

需要计算整体未命中的惩罚: \(Total = I_{penalty} + D_{penalty}\) 。

因此分别为:

\[\begin{aligned}
    t1 &= (6 + 1) \times (4\% + 6\% / 2) = 0.49 \\
    t2 &= (6 + 4) \times (2\% + 4\% / 2) = 0.4 \\ 
    t3 &= (6 + 4) \times (2\% + 3\% / 2) = 0.35
\end{aligned}\]

% Start Here

% End Here

\end{document}