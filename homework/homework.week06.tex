\documentclass[lang=cn,11pt,a4paper,cite=authoryear,twocolumn]{elegantpaper}

% 微分号
\newcommand{\dd}[1]{\mathrm{d}#1}
\newcommand{\pp}[1]{\partial{}#1}

\newcommand{\homep}[1]{\Large\textbf{Problem #1}}
\newcommand{\subhome}[1]{\large\textbf{SubProblem #1}}

% FT LT ZT
\newcommand{\ft}[1]{\mathscr{F}[#1]}
\newcommand{\fta}{\xrightarrow{\mathscr{F}}}
\newcommand{\lt}[1]{\mathscr{L}[#1]}
\newcommand{\lta}{\xrightarrow{\mathscr{L}}}
\newcommand{\zt}[1]{\mathscr{Z}[#1]}
\newcommand{\zta}{\xrightarrow{\mathscr{Z}}}

% 积分求和号

\newcommand{\dsum}{\displaystyle\sum}
\newcommand{\aint}{\int_{-\infty}^{+\infty}}

% 简易图片插入
\newcommand{\qfig}[3][nolabel]{
  \begin{figure}[!htb]
      \centering
      \includegraphics[width=0.6\textwidth]{#2}
      \caption{#3}
      \label{#1}
  \end{figure}
}

% 表格
\renewcommand\arraystretch{1.5}


% 日期

\lstset{
  mathescape=false
}

\title{计算机体系架构\quad 第六周作业}
\author{范云潜 18373486}
\institute{微电子学院 184111 班}
\date{\zhtoday}

\begin{document}

\maketitle

作业内容:B.2 B.4 B.6 B.11 B.18 B.21 

% \tableofcontents

\homep{B.2}

\[\begin{aligned}
    E &=((A \cdot B)+(A \cdot C)+(B \cdot C)) \cdot(\overline{A \cdot B \cdot C}) \\
    &= (AB + AC + BC) \cdot (A' + B' + C') \\
    &= (A'BC + AB'C + ABC') \\ 
    &=(A \cdot B \cdot \bar{C})+(A \cdot C \cdot \bar{B})+(B \cdot C \cdot \bar{A})
\end{aligned}\]

\homep{B.4}

\[A xor B = (A B') + (A' B)\]

\begin{table}[htb]
\caption{xor 真值表}
\centering
\label{tab:my-table}
\begin{tabular}{ll}\hline
\{AB\} & A xor B \\ \hline 
00     & 1       \\
01     & 0       \\
10     & 0       \\
11     & 1      \\\hline
\end{tabular}
\end{table}

\qfig[p1]{h6p1.png}{异或门实现}

\homep{B.6}

\[A' = (A \cdot 1)'\]

\[A \cdot B = ((A \cdot B)')'\]

\[A + B = (A' \cdot B')'\]

\homep{B.11}

\subhome{1}

\[(\sum (x_i = 0) == 1) = x_2'x_1x_0 + x_2x_1'x_0 + x_2x_1x_0'\]

\subhome{2}

\[\begin{aligned}
    (\sum (x_i = 0) \% 2 == 0) &= x_2'x_1'x_0 + x_2'x_1x_0'\\
     + x_2x_1'x_0' + x_2x_1x_0
\end{aligned}\]

\subhome{3}

\[\begin{aligned}
    (\mathtt{unsigned}  X < 4) &= x_2'x_1'x_0' + x_2'x_1'x_0 + \\
     & x_2'x_1x_0' + x_2'x_1x_0
\end{aligned}\]

\subhome{4}

\[\begin{aligned}
    (\mathtt{signed }\quad X < 0) &= x_2'
\end{aligned}\]

\homep{B.18}

\(\mathtt{FUNC1 }\) 是一个 MUX,在 \lstinline{S = 1} ,输出 \lstinline{I1} ,其他时刻输出 \lstinline{I0} 。

\lstinline{FUNC2} 是一个计数器,在复位时输出归零,其他时刻若是 \lstinline{ctl} 为一则输出加一,反之减一。

\homep{B.21}

\begin{lstlisting}
    module acc (
    input clk,
    input rst_a,
    input load,
    input [3:0] in,
    input [15:0] Load,
    output [15:0] out
);

reg [15:0] str;

assign out = str;

always @(posedge clk or rst_a) begin
    if (rst_a) begin
        str <= 0;
    end
    else begin
        if (load) begin
            str <= Load;
        end
        else begin
            str <= in + out;
        end
    end
end
    
endmodule
\end{lstlisting}
% Start Here

% End Here

\end{document}