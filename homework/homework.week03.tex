\documentclass[lang=cn,11pt,a4paper,cite=authoryear]{elegantpaper}

% 微分号
\newcommand{\dd}[1]{\mathrm{d}#1}
\newcommand{\pp}[1]{\partial{}#1}

\newcommand{\homep}[1]{\Large\textbf{Problem #1}}
\newcommand{\subhome}[1]{\large\textbf{SubProblem #1}}

% FT LT ZT
\newcommand{\ft}[1]{\mathscr{F}[#1]}
\newcommand{\fta}{\xrightarrow{\mathscr{F}}}
\newcommand{\lt}[1]{\mathscr{L}[#1]}
\newcommand{\lta}{\xrightarrow{\mathscr{L}}}
\newcommand{\zt}[1]{\mathscr{Z}[#1]}
\newcommand{\zta}{\xrightarrow{\mathscr{Z}}}

% 积分求和号

\newcommand{\dsum}{\displaystyle\sum}
\newcommand{\aint}{\int_{-\infty}^{+\infty}}

% 简易图片插入
\newcommand{\qfig}[3][nolabel]{
  \begin{figure}[!htb]
      \centering
      \includegraphics[width=0.6\textwidth]{#2}
      \caption{#3}
      \label{#1}
  \end{figure}
}

% 表格
\renewcommand\arraystretch{1.5}


% 日期

\lstset{
  mathescape=false
}

\title{计算机体系架构\quad 第三周作业}
\author{范云潜 18373486}
\institute{微电子学院 184111 班}
\date{\zhtoday}

\begin{document}

\maketitle

作业内容:3.1, 3.4, 3.7, 3.9, 3.14, 3.32, 3.37, 3.45

% \tableofcontents

\homep{3.1}

\lstinline{4096 = 0x1000 = 0b 00000000 00000000 00010000 00000000 }

\homep{3.4}

\lstinline{0b1111 1111 1111 1111 1111 1111 0000 0110 = 0xffffff06 = -250}

\homep{3.7}

\begin{lstlisting}
addu $t2, $t3, $0
# bltzal 
bgez $t3, ignore
sub $t2, $0, $t2
ignore:
\end{lstlisting}

\homep{3.9}

假设低位产生符号扩展,那么会变成负数,原本的 +x(low n-1 bit) 会变成 (x - (1<<n)) 那么将 \lstinline{A_upper_adjusted} 加 1 即可。若不产生,保持不变。

\homep{3.14}

\begin{lstlisting}
add $s1, $s0, $zero
sll $s0, $s0, 3
add $s1, $s0, $s1
\end{lstlisting}

\homep{3.32}


\begin{lstlisting}
#include <stdio.h>

int main()
{
    float num;
    int *nump;
    nump = (int *)(&num);
    scanf("%f", &num);
    printf("%x", *(nump));
    return 0;
}
\end{lstlisting}

\homep{3.37}

float 类型 0x41200000 = 0b 0100 0001 0010 0000 0000 0000 0000 0000, double 类型 0x4024000000000000 = 0b 0100 0000 0010  0100 0000 0000 0000 0000 0000 0000 0000 0000 0000 0000 0000 0000 。

% 10 = 8 + 2 = 1.010

% 10 + 10000000001

\homep{3.45}

将符号位放到 64 位的头部。
% Start Here

% End Here

\end{document}