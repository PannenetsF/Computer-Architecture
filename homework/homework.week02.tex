\documentclass[lang=cn,11pt,a4paper,cite=authoryear]{elegantpaper}

% 微分号
\newcommand{\dd}[1]{\mathrm{d}#1}
\newcommand{\pp}[1]{\partial{}#1}

\newcommand{\homep}[1]{\Large\textbf{Problem #1}}
\newcommand{\subhome}[1]{\large\textbf{SubProblem #1}}

% FT LT ZT
\newcommand{\ft}[1]{\mathscr{F}[#1]}
\newcommand{\fta}{\xrightarrow{\mathscr{F}}}
\newcommand{\lt}[1]{\mathscr{L}[#1]}
\newcommand{\lta}{\xrightarrow{\mathscr{L}}}
\newcommand{\zt}[1]{\mathscr{Z}[#1]}
\newcommand{\zta}{\xrightarrow{\mathscr{Z}}}

% 积分求和号

\newcommand{\dsum}{\displaystyle\sum}
\newcommand{\aint}{\int_{-\infty}^{+\infty}}

% 简易图片插入
\newcommand{\qfig}[3][nolabel]{
  \begin{figure}[!htb]
      \centering
      \includegraphics[width=0.6\textwidth]{#2}
      \caption{#3}
      \label{#1}
  \end{figure}
}

% 表格
\renewcommand\arraystretch{1.5}


% 日期

\lstset{
  mathescape=false
}

\title{计算机体系架构\quad 第二周作业}
\author{范云潜 18373486}
\institute{微电子学院 184111 班}
\date{\zhtoday}

\begin{document}

\maketitle

作业内容:2.10, 2.11, 2.13, 2.14; 2.15, 2.16, 2.21, 2.29;

% \tableofcontents

% Start Here

\homep{2.10}

\begin{lstlisting}
    .data
	.align 2
jtab: # jump table
	.word	L0, L1, L2, L3, L4 # exit

	.text 

.macro ret # return 
	jr   	$ra
.end_macro 
	

# f g h j i k
# 0 1 2 3 4 5

main:	# f-k -> s0-s5
	li	$t2, 4 # test code 
	li	$s5, 1 # test code 
	li	$s1, 1 # test code 
	li	$s2, 2 # test code 
	li	$s3, 3 # test code 
	li	$s4, 4 # test code 
	sub 	$t1, $t2, $s5 # t1 = 4-k 
	slt	$t0, $zero, $t1	#  0 < 4 - k should be 1
	beq 	$t0, $zero, L4 # check fail then exit
	slt 	$t0, $s5, $zero # k < 0 should be 0 
	bne	$t0, $zero, L4 # check fail then exit
	mul	$t1, $t2, $s5 # calculate the bias 
	la	$t0, jtab # t0 = addr_of_switch + 4 * k
	add 	$t1, $t0, $t1 # advance the pointer
	lw	$t0, ($t1) # load the dest memory addr 
	
	jalr	$t0 # start switch 
	j 	L4 # exit 
	
L0: 	add 	$s0, $s3, $s4 # case 0
	ret 
L1: 	add	$s0, $s1, $s2 # case 1
	ret 
L2: 	sub	$s0, $s1, $s2 # case 2
	ret
L3:	sub 	$s0, $s3, $s4 # case 3
	ret 
L4: 	li      $v0, 10	# exit
	syscall			

	
\end{lstlisting}    

\homep{2.11}

\subhome{a}

\begin{lstlisting}
if (k == 0) {
    f = i + j;
}else if(k == 1) {
    f = g + h;
}else if (k == 2) {
    f = g - h;
}else if (k == 3) {
    f = i - j;
}else {
    return 0; // check failed
}
\end{lstlisting}

\subhome{b}


\begin{lstlisting}
	.data
	.align 2
jtab:
	.word	L0, L1, L2, L3, L4 # exit

	.text 

.macro ret
	jr   	$ra
.end_macro 

.macro exit
	li      $v0, 10	# exit
	syscall
.end_macro
# f g h j i k
# 0 1 2 3 4 5

main:	# f-k -> s0-s5
	li	$t2, 4 # test code 
	li	$s5, 1 # test code 
	li	$s1, 1 # test code 
	li	$s2, 2 # test code 
	li	$s3, 3 # test code 
	li	$s4, 4 # test code 
	beq 	$t1, $zero, L0 # cmp with 0
	subi	$t1, $s5, 1
 	beq 	$t1, $zero, L1 # cmp with 1
	subi	$t1, $s5, 2
	beq 	$t1, $zero, L2 # cmp with 2
	subi	$t1, $s5, 3
	beq 	$t1, $zero, L3 # cmp with 3
	j 	L4 # else: return

	
L0: 	add 	$s0, $s3, $s4 # case 0
	exit 
L1: 	add	$s0, $s1, $s2 # case 1
	exit
L2: 	sub	$s0, $s1, $s2 # case 2
	exit
L3:	sub 	$s0, $s3, $s4 # case 3
	exit
L4: 				
	exit
	
\end{lstlisting}    

\subhome{3}

对于跳转表,算数类:6 ,传输类:3 ,分支类:2,跳转类:3 ,共 17.2 clk 。

对于 if-else ,算数类:4 ,传输类:1 ,分支类:4,跳转类:1 ,共 13.4 clk 。

但是对于更多分支类型的表达式,跳转表会更加迅速。

\homep{2.13}

如 \figref{213}。

\qfig[213]{h2p1.png}{循环流程图}

\homep{2.14}


\begin{lstlisting}
    .data
	.align 2
jtab: # jump table
	.word	Loop, Exit

	.text 
.macro ret # return 
	jr   	$ra
.end_macro 
    
.macro exit
	li      $v0, 10	# exit
	syscall
.end_macro

# s3 -> i
# s6 -> save 
# s5 -> k

main:	

Loop:
    sll     $t1, $s3, 2 # t1 = i * 4 
    add     $t1, $t1, $s6
    lw      $t0, 0($t1)
	sub 	$t1, $t2, $s5 # t1 = 4-k 
	slt	$t0, $zero, $t1	#  0 < 4 - k should be 1
	beq 	$t0, $zero, L4 # check fail then exit
	slt 	$t0, $s5, $zero # k < 0 should be 0 
	bne	$t0, $zero, L4 # check fail then exit
	mul	$t1, $t2, $s5 # calculate the bias 
	la	$t0, jtab # t0 = addr_of_switch + 4 * k
	add 	$t1, $t0, $t1 # advance the pointer
	lw	$t0, ($t1) # load the dest memory addr 
	
	jalr	$t0 # start switch 
	j 	L4 # exit 
	
L0: 	add 	$s0, $s3, $s4 # case 0
	ret 
L1: 	add	$s0, $s1, $s2 # case 1
	ret 
L2: 	sub	$s0, $s1, $s2 # case 2
	ret
L3:	sub 	$s0, $s3, $s4 # case 3
	ret 
L4: 	li      $v0, 10	# exit
	syscall			

	
\end{lstlisting}    

\homep{2.15}


\begin{lstlisting}
		.text

.macro ret # return 
	jr   	$ra
.end_macro 

# int i in $s0

set_array:
	# allocate space for: fp/ra/array[10]/num = 4 * (1+1+10+1) = 52
	addi 	$sp, $sp, -52 
	# store fp, ra, num for the caller,
	# num is the only args, in a0
	sw	$fp, 48($sp)
	sw	$ra, 44($sp)
	sw	$a0, 40($sp)
	# init the fp for the stack
	addi	$fp, $sp, 48 
	# set i = 0, max = 10
	add 	$s0, $zero, $zero 
	addi 	$t0, $zero, 10 
	
	
loop:	
	# set bias as 4*i to index array 
	sll 	$t1, $s0, 2
	add 	$t1, $t1, $sp 
	# i++ 
	addi 	$s0, $s0, 1 
	# set num and i as args
	add 	$a0, $a0, $zero 
	add 	$a1, $s0, $zero
	jal 	compare
	sw 	$v0, ($t1) 
	bne 	$s0, $t0, loop
	
	# loop end, then restore the stack for caller
	lw	$fp, 48($sp)
	lw	$ra, 44($sp)
	lw	$a0, 40($sp)
	addi 	$sp, $sp, 52
	ret 

compare:
	# allocate for fp/ra
	addi 	$sp, $sp, -8
	sw	$fp, 4($sp)
	sw	$ra, ($sp)
	
	jal	Sub
	# if (v0 < 0)==1 < 1 == 0, then return 0; else return 1
	slt	$v0, $v0, $zero
	slti 	$v0, $v0, 1
	
	# restore the frame for caller
	lw 	$fp, 4($sp)
	lw 	$ra, ($sp)
	addi 	$sp, $sp, 8
	ret
	

Sub: 
	# allocate for fp/ra
	addi 	$sp, $sp, -8
	sw	$fp, 4($sp)
	sw	$ra, ($sp)
	
	sub	$v0, $a0, $a1
	
	# restore the frame for caller
	lw 	$fp, 4($sp)
	lw 	$ra, ($sp)
	addi 	$sp, $sp, 8
	ret
	
\end{lstlisting}

栈的示意图如 \figref{223}

\qfig[223]{h2p2.png}{stack 示意图}

\homep{2.16}


\begin{lstlisting}
# n stored in $a0

	.data
msg: .asciiz "the ans is "

	.text

	addi 	$s0, $zero, 1
	# li	$a0, 7
	# la	$ra, prt
fib:	
	addi	$sp, $sp, -12
	# push ra and n to stack
	sw 	$ra, 8($sp)
	sw 	$a0, 4($sp)
	# if n == 0, return 0; 
	beq	$a0, $zero, n0
	# if n == 1, return 1;
	beq	$a0, $s0, n1
	# else fib(n-1)
	addi 	$a0, $a0, -1
	jal	fib
	# t0 = fib(n-1)
	add 	$t0, $v0, $zero
	sw 	$t0, ($sp) 
	addi 	$a0, $a0, -1
	jal	fib
	# v0 = fib(n-1) + fib(n-2)
	# store the answer
	lw	$t0, ($sp)
	add 	$v0, $t0, $v0
	# pop ra and n to stack 
done:	lw	$ra, 8($sp)
	lw	$a0, 4($sp)
	addi 	$sp, $sp, 12
	
	# return to the caller
	jr	$ra
	
	

n0:	add	$v0, $zero, $zero
	j	done 
n1: 	addi	$v0, $zero, 1
	j 	done
prt:	add 	$a0, $zero, $v0
	li      $v0, 1
	syscall 
	
	
\end{lstlisting}





\homep{2.21}

其 C 原型为 

\begin{lstlisting}

int str2int(char * str) {
    const char ascii_0 = '0';
    const char ascii_9 = '9';
    const int digit = 10;
    int ret = 0;

    while(1) {
		char now = *str;
		str++;
        if (now != '\0') {
            if (now <= ascii_9 && now >= ascii_0) {
                ret *= 10;
                ret += now - ascii_0;
            }
            else {
                return -1;
            }
        }else {
            return ret;
        }
    }
} 

\end{lstlisting}

对此进行翻译

\begin{lstlisting}
	.data
str: 	.asciiz	"1-3"

	.text
	la	$ra, end
	la	$a0, str
#	j 	puts
str2int:
	addi	$t0, $zero, 47 # 47 is the '0' - 1
	addi 	$t1, $zero, 58 # 58 is the '9' + 1
	addi 	$t2, $zero, 10 # set the number mult
	addi	$t4, $zero, 48 # 48 is the '0', 
	add 	$v0, $zero, $zero # set ret as 0
loop:
	lb	$t3, ($a0) # now = *str
	addi	$a0, $a0, 1 # str++
	beq 	$t3, $zero, ret # if end, then return 
	slt	$t5, $t3, $t1 # if now <= '9'
	slt 	$t6, $t0, $t3 # if '0' <= now
	and	$t5, $t5, $t6 # and the condition
	bne	$t5, 1, err # fail then return -1
	mul	$v0, $t2, $v0 # ret *= 10
	sub	$t3, $t3, $t4
	add	$v0, $v0, $t3
	j	loop
	
err:
	addi	$v0, $zero, -1
	j	ret

ret:	
	jr 	$ra
	
end: 	
	add	$a0, $v0, $zero
	j 	puti
	
puti:
        li      $v0, 1          # specify Print Integer service
        syscall                 # Print it
        jr      $ra             # Return



puts:
        li      $v0, 4          # specify Print String service
        syscall                 # Print it
        jr      $ra             # Return


\end{lstlisting}

\homep{2.29}

计算 \lstinline{a * b + 100} 。

\begin{lstlisting}
		add	$t0, $zero, $zero # set $t0 to 0
loop:	beq 	$a1, $zero, finish # if b == 0, then goto finish 
		add	$t0, $t0, $a0 # else t0 += a
		sub	$a1, $a1, 1 # b--
		j 	loop
finish	addi	$t0, $t0, 100 # t0 += 100
		add	$v0, $t0, $zero # return t0
\end{lstlisting}
% End Here


\end{document}