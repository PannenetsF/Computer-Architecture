\documentclass[lang=cn,11pt,a4paper,cite=authoryear]{elegantpaper}

% 微分号
\newcommand{\dd}[1]{\mathrm{d}#1}
\newcommand{\pp}[1]{\partial{}#1}

\newcommand{\homep}[1]{\Large\textbf{Problem #1}}
\newcommand{\subhome}[1]{\large\textbf{SubProblem #1}}

% FT LT ZT
\newcommand{\ft}[1]{\mathscr{F}[#1]}
\newcommand{\fta}{\xrightarrow{\mathscr{F}}}
\newcommand{\lt}[1]{\mathscr{L}[#1]}
\newcommand{\lta}{\xrightarrow{\mathscr{L}}}
\newcommand{\zt}[1]{\mathscr{Z}[#1]}
\newcommand{\zta}{\xrightarrow{\mathscr{Z}}}

% 积分求和号

\newcommand{\dsum}{\displaystyle\sum}
\newcommand{\aint}{\int_{-\infty}^{+\infty}}

% 简易图片插入
\newcommand{\qfig}[3][nolabel]{
  \begin{figure}[!htb]
      \centering
      \includegraphics[width=0.6\textwidth]{#2}
      \caption{#3}
      \label{#1}
  \end{figure}
}

% 表格
\renewcommand\arraystretch{1.5}


% 日期

\lstset{
  mathescape=false
}

\title{计算机体系架构\quad Lab02}
\author{范云潜 18373486}
\institute{微电子学院 184111 班}
\date{\zhtoday}

\begin{document}

\maketitle

% 作业内容:

% \tableofcontents

% Start Here

\homep{ex1}

对于一个 32-bit 的整数 \lstinline{x},加法发生溢出的边界是 \lstinline{~x},此时恰好得到 \lstinline{~0x0} ,若是加数大于此边界(无符号类型)即可判断发生了溢出。

\lstinputlisting[caption={ex1 solution}]{lab02/ex1.asm}

\homep{ex2}

对于一个浮点数,当其越来越大,每两个数字之间的间距也会越来越大,那么我们首先挑选一个很大的浮点数\footnote{通过 C 进行转换成浮点数} \{0b 0 | 111 1111 0 | 000 0000 0000 0000 0000 = 170141183460469231731687303715884105728.000000 = 0x7f000000\} 。通过 C 和 MARS 进行实验,发现的确如此。

\homep{ex3}

此处需要找到一个\textbf{最小的}浮点数,因此需要仔细考虑浮点数的间隔下界大于 1 的临界点,此时作为尾数域的最小变化会引起整个浮点数增加 2 ,那么指数是 24 = 0x18 ,那么指数域为 24+127 = 0x97 ,那么位表示为 \{0b 0 | 100 1011 1 | 000 0000 0000 0000 0000 = 16777216.000000 = 0x4b800000\} 。若是尝试 \{0x4b800000 - 1= 0x4b7fffff = 16777215.000000\} 则会失败,验证完毕。

\homep{ex4}

事实上浮点数的不可交换原因之一是浮点数的特殊值也就是 \(\infty\) 和 NaN 。如果前两个值相加会溢出,而第三个值的提前出现会防止这种情况的出现。这样的集合可以表示为

\[\{a, b, c | a + b \text{ overflow } \&\&  |a+c|< |a| \text{ thus } (a+c)+b \text{ don't overflow}\}\]

进行举例,最大的规格化值 \{0b 0 1111 1110 1111 1111 1111 1111 1111 111 \} ,它的一半 \{0b 0 1111 1101 1111 1111 1111 1111 1111 111\} , 它的相反数  \{0b 1 1111 1111 1110 1111 1111 1111 1111 111 \} 分别作为 a, b, c 。

\lstinputlisting[caption={ex4 solution-1}]{lab02/ex4.c}

% \lstinputlisting[caption={ex4 solution}]{lab02/ex4.c}
输出为 

\begin{lstlisting}
    340282346638528859811704183484516925440.000000
    170141173319264429905852091742258462720.000000
    -170141173319264429905852091742258462720.000000
    inf
    340282346638528859811704183484516925440.000000    
\end{lstlisting}
   
另一方面的原因还包括,对于同一符号类型的数据,由于阶数不同,相邻浮点数的间隔也不同,若是求和时顺序使得被加数在间隔内,则不会引起变化。定义浮点数 \(a\) 的阶数下,尾数变化的最小单位为 \(\delta_a\) 。那么 \((a+b)+c \neq a+(b+c)\) 可以表述为

\[\{a, b, c| a \geq \delta_b \&\& a < \delta_{b+c}\}\]

举例,\(b\) 和 \(c\) 取间隔为 2 的浮点数 \{0b 0 | 100 1011 1 | 000 0000 0000 0000 0000 = 16777216.000000 = 0x4b800000\} ,\(a\)  取浮点数 2.1 。

\lstinputlisting[caption={ex4 solution-2}]{lab02/ex4_2.c}

输出为 

\begin{lstlisting}
    40066666
    33554432.000000
    33554436.000000
\end{lstlisting}

进一步的,需要进行 MIPS 的程序设计,请见 p2\_1.s 。


\lstinputlisting[caption={ex4 solution-2 mips}]{lab02/p2_1.s}

输出为 

\begin{lstlisting}
    1.6777216E7
    2.1
    (a+b)+c is 3.3554432E7
    a+(b+c) is 3.3554436E7

-- program is finished running (dropped off bottom) --
\end{lstlisting}


% End Here

\end{document}