\documentclass[lang=cn,11pt,a4paper,cite=authoryear]{elegantpaper}

\input{needed.tex}

\title{计算机体系架构\quad 第一周作业}
\author{范云潜 18373486}
\institute{微电子学院 184111 班}
\date{\zhtoday}

\begin{document}

\maketitle

作业内容: Lab01

\tableofcontents



% Start Here

\homep{Exercise 1}

\subhome{Directives}

\lstinline{.data} 表明使用了一个小节专门用于存储程序的数据。(.data       Subsequent items stored in Data segment at next available address)

\lstinline{.word} 表明定义的数据是以一个字( word )为单位的,在这里定义了一个字大小对应的内存位置,存储了数字 9 ,并且可以使用标签 \lstinline{n} 来引用对应的内存位置。(.word       Store the listed value(s) as 32 bit words on word boundary)

\lstinline{.text} 表示代码段开始。(.text       Subsequent items (instructions) stored in Text segment at next available address)

\subhome{Breakpoint}


% http://courses.missouristate.edu/kenvollmar/mars/help/syscallhelp.html
% End Here

\end{document}