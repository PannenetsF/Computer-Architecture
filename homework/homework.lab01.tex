\documentclass[lang=cn,11pt,a4paper,cite=authoryear]{elegantpaper}

% 微分号
\newcommand{\dd}[1]{\mathrm{d}#1}
\newcommand{\pp}[1]{\partial{}#1}

\newcommand{\homep}[1]{\Large\textbf{Problem #1}}
\newcommand{\subhome}[1]{\large\textbf{SubProblem #1}}

% FT LT ZT
\newcommand{\ft}[1]{\mathscr{F}[#1]}
\newcommand{\fta}{\xrightarrow{\mathscr{F}}}
\newcommand{\lt}[1]{\mathscr{L}[#1]}
\newcommand{\lta}{\xrightarrow{\mathscr{L}}}
\newcommand{\zt}[1]{\mathscr{Z}[#1]}
\newcommand{\zta}{\xrightarrow{\mathscr{Z}}}

% 积分求和号

\newcommand{\dsum}{\displaystyle\sum}
\newcommand{\aint}{\int_{-\infty}^{+\infty}}

% 简易图片插入
\newcommand{\qfig}[3][nolabel]{
  \begin{figure}[!htb]
      \centering
      \includegraphics[width=0.6\textwidth]{#2}
      \caption{#3}
      \label{#1}
  \end{figure}
}

% 表格
\renewcommand\arraystretch{1.5}


% 日期

\lstset{
  mathescape=false
}

\title{计算机体系架构\quad 第一周作业}
\author{范云潜 18373486}
\institute{微电子学院 184111 班}
\date{\zhtoday}

\begin{document}

\maketitle

作业内容: Lab01

\tableofcontents



% Start Here

\homep{Exercise 1}

\subhome{Directives}

\lstinline{.data} 表明使用了一个小节专门用于存储程序的数据。(.data       Subsequent items stored in Data segment at next available address)

\lstinline{.word} 表明定义的数据是以一个字( word )为单位的,在这里定义了一个字大小对应的内存位置,存储了数字 9 ,并且可以使用标签 \lstinline{n} 来引用对应的内存位置。(.word       Store the listed value(s) as 32 bit words on word boundary)

\lstinline{.text} 表示代码段开始。(.text       Subsequent items (instructions) stored in Text segment at next available address)

\subhome{Breakpoint}


% http://courses.missouristate.edu/kenvollmar/mars/help/syscallhelp.html
% End Here

\end{document}