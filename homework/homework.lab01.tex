\documentclass[lang=cn,11pt,a4paper,cite=authoryear]{elegantpaper}

% 微分号
\newcommand{\dd}[1]{\mathrm{d}#1}
\newcommand{\pp}[1]{\partial{}#1}

\newcommand{\homep}[1]{\Large\textbf{Problem #1}}
\newcommand{\subhome}[1]{\large\textbf{SubProblem #1}}

% FT LT ZT
\newcommand{\ft}[1]{\mathscr{F}[#1]}
\newcommand{\fta}{\xrightarrow{\mathscr{F}}}
\newcommand{\lt}[1]{\mathscr{L}[#1]}
\newcommand{\lta}{\xrightarrow{\mathscr{L}}}
\newcommand{\zt}[1]{\mathscr{Z}[#1]}
\newcommand{\zta}{\xrightarrow{\mathscr{Z}}}

% 积分求和号

\newcommand{\dsum}{\displaystyle\sum}
\newcommand{\aint}{\int_{-\infty}^{+\infty}}

% 简易图片插入
\newcommand{\qfig}[3][nolabel]{
  \begin{figure}[!htb]
      \centering
      \includegraphics[width=0.6\textwidth]{#2}
      \caption{#3}
      \label{#1}
  \end{figure}
}

% 表格
\renewcommand\arraystretch{1.5}


% 日期

\lstset{
  mathescape=false
}

\title{计算机体系架构\quad 第一周作业}
\author{范云潜 18373486}
\institute{微电子学院 184111 班}
\date{\zhtoday}

\begin{document}

\maketitle

作业内容: Lab01

\tableofcontents



% Start Here

\homep{Exercise 1}

\subhome{Directives}

\lstinline{.data} 表明使用了一个小节专门用于存储程序的数据。(.data       Subsequent items stored in Data segment at next available address)

\lstinline{.word} 表明定义的数据是以一个字( word )为单位的,在这里定义了一个字大小对应的内存位置,存储了数字 9 ,并且可以使用标签 \lstinline{n} 来引用对应的内存位置。(.word       Store the listed value(s) as 32 bit words on word boundary)

\lstinline{.text} 表示代码段开始。(.text       Subsequent items (instructions) stored in Text segment at next available address)

\subhome{Breakpoint}

在汇编之后,可以在执行顺序旁进行断点添加,类似于 gdb 的 \lstinline{b} ,如\figref{1}。直接运行会运行到下一个断点或直接完成程序,可以使用 Step 模式进行单条指令执行。

\qfig[1]{l1p1.png}{断点添加界面}

\subhome{Regs}

寄存器的值可以从侧面寄存器面板直接获得,如\figref{2},双击即可强制修改。


\qfig[2]{l1p2.png}{寄存器面板}

\subhome{Mem}

根据 Data Segment 面板,n 存储到 \lstinline{0x10010000} 。

\subhome{syscall}

对于模拟器提供的 \lstinline{syscall} 系列的函数,有着详细的文档说明\footnote{http://courses.missouristate.edu/kenvollmar/mars/help/syscallhelp.html},这里使用的是 1 和 10 号 syscall ,分别是 \lstinline{print_int} 与 \lstinline{exit} 。

为了使用 syscall 需要在 \lstinline{$vi} 中写入调用标号,之后使用 \lstinline{syscall} 。

\homep{Exercise 2}

实际上这个代码段只用到了 \lstinline{add} 指令,分别用来加载寄存器内的值以及进行算数运算。为了验证代码的正确性,使用 python 完成了一个 \lstinline{gold} 函数用于模拟行为,请见 lab1\_ex2.py。

\homep{Exercise 3}

提炼出的 Bugs 如下:

\begin{enumerate}
    \item 未对计数器 \lstinline{$v0} 进行初始化而直接进行计数
    \item 指向一个 word 的指针应当以 4 bytes 为单位,而不是 1 byte
    \item 由于最后一个数据为 0 ,不应计数
\end{enumerate}

\homep{Exercise 4}

循环段落在,并且为每一行添加了注释进行说明

\begin{lstlisting}
$L3:
    sw      $3,0($4)    # store the read word to 'dest'
    addiu   $6,$6,1     # add 1 to the counter of loop
    lw      $3,4($2)    # read next word of 'source'
    addiu   $4,$4,4     # advance the pointer of 'dest' by a word 
    addiu   $2,$2,4     # advance the pointer of 'source' by a word 
    bne     $3,$0,$L3   # loop condition: if read a '0' then stop
    nop
\end{lstlisting}    


source 在 

\begin{lstlisting}
source:
    .word   3
    .word   1
    .word   4
    .word   1
    .word   5
    .word   9
    .word   0
    .ident  "GCC: (Ubuntu 7.5.0-3ubuntu1~18.04) 7.5.0"

\end{lstlisting}

dest 被定义为 40 个字节长,4 字节为单位的内存: \lstinline{.comm   dest,40,4}



% End Here

\end{document}