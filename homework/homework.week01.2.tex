\documentclass[lang=cn,11pt,a4paper,cite=authoryear]{elegantpaper}

% 微分号
\newcommand{\dd}[1]{\mathrm{d}#1}
\newcommand{\pp}[1]{\partial{}#1}

\newcommand{\homep}[1]{\Large\textbf{Problem #1}}
\newcommand{\subhome}[1]{\large\textbf{SubProblem #1}}

% FT LT ZT
\newcommand{\ft}[1]{\mathscr{F}[#1]}
\newcommand{\fta}{\xrightarrow{\mathscr{F}}}
\newcommand{\lt}[1]{\mathscr{L}[#1]}
\newcommand{\lta}{\xrightarrow{\mathscr{L}}}
\newcommand{\zt}[1]{\mathscr{Z}[#1]}
\newcommand{\zta}{\xrightarrow{\mathscr{Z}}}

% 积分求和号

\newcommand{\dsum}{\displaystyle\sum}
\newcommand{\aint}{\int_{-\infty}^{+\infty}}

% 简易图片插入
\newcommand{\qfig}[3][nolabel]{
  \begin{figure}[!htb]
      \centering
      \includegraphics[width=0.6\textwidth]{#2}
      \caption{#3}
      \label{#1}
  \end{figure}
}

% 表格
\renewcommand\arraystretch{1.5}


% 日期

\lstset{
  mathescape=false
}

\title{计算机体系架构\quad 第二周作业}
\author{范云潜 18373486}
\institute{微电子学院 184111 班}
\date{\zhtoday}

\begin{document}

\maketitle

作业内容: 3.2, 3.4, 3.6, 3.8;2.1, 2.2, 2.5, 2.6;

% \tableofcontents

% Start Here

\homep{3.2}

根据补码取负值的原则 \(\mathtt{2047} = \mathtt{0x000007ff}\) ,
\(\mathtt{-2047} = \mathtt{0xfffff801}\) 。

\homep{3.4}

表示为十六进制为 \(\mathtt{0xffffff06}\) ,以补码换算到十进制 \(\mathtt{-250}\) 。

\homep{3.6}

表示为十六进制为 \(\mathtt{0x7fffffef}\) ,以补码换算到十进制 \(\mathtt{2147483631}\) 。

\homep{3.8}

对于 Harry ,可以举反例 \(\mathtt{0b1111 * 0b0110} = \mathtt{0b1011010}\) 。

对于 David ,可以举反例 \lstinline{36 + 6 = 0b100100 + 0b110 = 0b101010} ,仅有 3 个 ‘1’ 。

\homep{2.1}

原始代码

\begin{lstlisting}
        addi    $v0, $zero, 0 # the increase is after init
        # I: 001000 00000 00010  0000000000000000
loop:   lw      $v1, 0($a1)
        # I: 100011 00101 00011  0000000000000000
        sw      $v1, 0($a1)
        # I: 000010 00101 00011  0000000000000000
        addi    $a0, $a0,   4
        # I: 001000 00100 00100  0000000000000100
        addi    $a1, $a1,   4
        # I: 001000 00101 00101  0000000000000100
        beq     $v1, $zero, loop # loop condition
        # I: 000100 00000 00011  1111111111111011 (-5 * 4) 0101 1011
\end{lstlisting}

按照汇编格式,写成 指令 / 对应十进制数的形式。

\begin{lstlisting}
0x20020000 / 537001984
0x8ca30000 / 2359492608
0xaca30000 / 2896363520
0x20840004 / 545521668
0x20a50004 / 547684356
0x1060fffb / 274792443
\end{lstlisting}    

bug 修改后:

\begin{lstlisting}
        addi    $v0, $zero, -1 # the increase is after init
loop:   lw      $v1, 0($a1)
        sw      $v1, 0($a1)
        addi    $a0, $a0,   4
        addi    $a1, $a1,   4
        addi    $v0, $v0,   1 # add 1 to the cnt 
        bne     $v1, $zero, loop # loop condition

\end{lstlisting}

% 对 \lstinline{addi},

\homep{2.2}

\(
\mathtt{0x7fff fffa} = \mathtt{0b 0111 1111 1111 1111 1111 1111 1111 1010} = \mathtt{2147483642}
\)


\homep{2.5}

\begin{lstlisting}
# assign t0=a, s0=b, s1=c, s2=d
		.data
mask:	
		.word	0xfffff83f
		.text
start:	
		lw	 $t0, mask	# a=0xfffff83f   1000 0011 1111
		lw	 $s0, shifter	# 
		and  $s0, $s0, $t0      # apply the mask to b
		andi $s2, $s2, 0x1f     # select low-5 bit of d
		sll  $s2, $s2, 6        
		or   $s0, $s0, $s2      # fill the masked bits of b
		sw   $s0, shifter       # change the shamt
shifter:
		sll  $s0, $s1, 0 
\end{lstlisting}

整体流程如下:

\begin{enumerate}
    \item 加载一个 MASK ,可以通过与操作清空 R 型指令 \lstinline{sll} 的 shamt 区段
    \item 通过另一个 MASK 选择需要移位的位数,也就是 \lstinline{$s2} 的低 5 位
    \item 将这个位数加载到 shamt 区段再执行指令即可完成自定义位数的移位操作
\end{enumerate}

但是我们可以认识到,这是一种不显式的操作,无法通过阅读单独的一条指令确定其功能,这样的写法使得指令存在耦合,增大了程序调试与维护的难度。

\homep{2.6}

为了不出现位扩展,采用逻辑移位操作。

\begin{lstlisting}
sll $t3, $t3, 10
srl $t1, $t3, 15
\end{lstlisting}    

\section*{词汇}

explicit 显式的





% End Here

\end{document}