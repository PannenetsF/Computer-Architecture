\ifx\mainclass\undefined
\documentclass[cn,11pt,chinese,black,simple]{../elegantbook}
% 微分号
\newcommand{\dd}[1]{\mathrm{d}#1}
\newcommand{\pp}[1]{\partial{}#1}

\newcommand{\homep}[1]{\Large\textbf{Problem #1}}
\newcommand{\subhome}[1]{\large\textbf{SubProblem #1}}

% FT LT ZT
\newcommand{\ft}[1]{\mathscr{F}[#1]}
\newcommand{\fta}{\xrightarrow{\mathscr{F}}}
\newcommand{\lt}[1]{\mathscr{L}[#1]}
\newcommand{\lta}{\xrightarrow{\mathscr{L}}}
\newcommand{\zt}[1]{\mathscr{Z}[#1]}
\newcommand{\zta}{\xrightarrow{\mathscr{Z}}}

% 积分求和号

\newcommand{\dsum}{\displaystyle\sum}
\newcommand{\aint}{\int_{-\infty}^{+\infty}}

% 简易图片插入
\newcommand{\qfig}[3][nolabel]{
  \begin{figure}[!htb]
      \centering
      \includegraphics[width=0.6\textwidth]{#2}
      \caption{#3}
      \label{#1}
  \end{figure}
}

% 表格
\renewcommand\arraystretch{1.5}


% 日期

\lstset{
  mathescape=false
}
\begin{document}
\fi 
\def\chapname{00intro}

% Start Here
\chapter{课程介绍}

任课教师:成元庆

联系方式:yuanqing@ieee.org 18911370169

课程助教:倪嘉诚

\section{课程简介}

计算机体系架构是连接软硬件的一门课程,
向下连接集成电路设计,
向上联系计算机高层的软件、编译器、操作系统等。

近年来,出现了几种新型的 CPU 架构,但是整体来说,设计思路没有大的变化,本质来讲,目前的计算架构采用的仍然是冯 \(\cdot\) 诺依曼架构。

本课程采用 MIPS 架构进行讲解,与 RISC-V 以及 ARM 等架构几乎同源,我国的龙芯项目也是采用的本指令集。教材采用 《 Computer Organization and Design 》 。

本课程的课件采用英文编写,因为大量的术语尚未有确切的中文翻译,并且大量的文献都是英文编写。


本课程有一定的前置课程,采用 C 语言进行教授,需要使用到部分数据结构的知识以及一些数字逻辑设计。


\section{电脑是智能的吗?}

对于编程者,是不需要考虑具体的实现的,如考虑数据管理、函数调用等操作,编程者可以按照思维设计软件的结构。
但是对于最底层的电路来说,实际支持的可能只有与或非等基本的操作,只支持电路的二值化,硬件并不会自动的进行上述的这些工作。
由操作系统这种底层的软件进行一系列的复杂调度,来作为沟通软硬件的接口。
层次划分特点,为计算机体系提供了广泛的抽象,这种抽象可以使不同层次的使用者不需要了解过于细节的实现。

软件的执行需要最终通过汇编器翻译成机器码,执行机器码需要操作系统对硬件的调用。

\section{计算机表示的层次}

高等语言会通过 Compile 转换为汇编语言,再通过 Assemble 转会为机器码,准备在机器上执行。
硬件的架构也是一种抽象,如各种模块 ALU、RegFile等,最终需要落实到不同的逻辑门上。

\section{计算机的结构}

计算机的结构基本如下:

\begin{itemize}
    \item 处理器
    \begin{itemize}
        \item 控制部分
        \item 数据通路
    \end{itemize}
    \item 主存
    \item 输入
    \item 输出
\end{itemize}

\section{集成电路}

对于裸片( Bare Die )存在不同的 CMOS 工艺,最终需要封装到 PCB 板上。 PCB 一般是树脂或者塑料的衬板。
到目前来说,集成电路的发展基本吻合摩尔定律 (2X/18 Months)。单位处理器的速度以 1.20-1.52 倍每年的倍率上升,但是随着单核处理器的某些瓶颈出现,速度逐渐放缓。

DRAM 也就是动态随机存储器,用于计算机的主存从 1980 年代到如今存储密度已经上升了超过 8000 倍。

\section{教学方向}

计算机体系的速度限制是什么?

基本的教学方向:

\begin{itemize}
    \item 计算机层级的抽象
    \item 五个基本的计算机组成部分
\end{itemize}

具体来说

\begin{markdown}

\begin{itemize}
    \item 计算机的机器表示
    \begin{itemize}
        \item 数字表示
        \item 汇编语言
        \item 编译与汇编    
    \end{itemize} 
    \item 处理器与硬件
    \begin{itemize}
        \item 逻辑电路设计
        \item  CPU 组织
        \item  流水线
    \end{itemize}
    \item 存储组织
    \begin{itemize}
        \item 缓存
        \item 虚拟内存
    \end{itemize}
    \item 输入输出
    \begin{itemize}
        \item 中断
        \item 硬盘与网络
    \end{itemize}
    \item 其他
    \begin{itemize}
        \item 性能
        \item 虚拟化
        \item 并行化
    \end{itemize}
\end{itemize}

- 
    - 
    - 虚拟内存
- 输入输出
    - 中断
    - 硬盘与网络
- 其他
    - 性能
    - 虚拟化
    - 并行化
        - sa
\end{markdown}

\section*{单词}

brawn 肌肉

anatomy 解剖

chassis 底盘

dramatic 急剧的
% End Here

\let\chapname\undefined
\ifx\mainclass\undefined
\end{document}
\fi 