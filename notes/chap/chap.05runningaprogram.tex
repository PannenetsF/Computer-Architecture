\ifx\mainclass\undefined
\documentclass[cn,11pt,chinese,black,simple]{../elegantbook}
\input{../needed.tex}
\begin{document}
\fi 
\def\chapname{05runningaprogram}

% Start Here
\chapter{运行程序}

\section{解释与编译}

直接将源代码解释为可执行的文件。如 Java 转换为 Byte Code 。为了提高解释性程序的性能,通过统计进行优化,可以将自己编译成对应平台的二进制码。解释型的语言有跨平台的特性。

编译性的程序报错更多,错误更难定位与调试。而优化后的代码会产生源码的不对应。最终产生的可执行文件相当小。

对于汇编,需要进行扫描来确定不确定的那些地址。对于 Object 中的绝对地址,只能通过链接器进行确定,具体由符号表( Symbol Table )实现。一些不可到达的大距离跳转,以及改变的被分配地址,通过 Relocation Table 重定位得到绝对地址。

对于可执行文件需要有头部信息,代码段,数据段,调试信息。

Java 使用解释器与翻译器的原因是跨平台支持。

\section{链接与加载}

链接可以提高编译的效率和可维护性。

为了解决函数的引用问题,从源代码、库进行搜索。

静态链接库可以方便的运行,但是打包了所有的库,体积较大。动态库虽然需要反复的加载,但是速度一般更快。

加载器加载库并运行。会加载命令行参数、清空寄存器、跳转到开始代码加载参数之后调用主函数。
% End Here

\let\chapname\undefined
\ifx\mainclass\undefined
\end{document}
\fi 