\ifx\mainclass\undefined
\documentclass[cn,11pt,chinese,black,simple]{../elegantbook}
% 微分号
\newcommand{\dd}[1]{\mathrm{d}#1}
\newcommand{\pp}[1]{\partial{}#1}

\newcommand{\homep}[1]{\Large\textbf{Problem #1}}
\newcommand{\subhome}[1]{\large\textbf{SubProblem #1}}

% FT LT ZT
\newcommand{\ft}[1]{\mathscr{F}[#1]}
\newcommand{\fta}{\xrightarrow{\mathscr{F}}}
\newcommand{\lt}[1]{\mathscr{L}[#1]}
\newcommand{\lta}{\xrightarrow{\mathscr{L}}}
\newcommand{\zt}[1]{\mathscr{Z}[#1]}
\newcommand{\zta}{\xrightarrow{\mathscr{Z}}}

% 积分求和号

\newcommand{\dsum}{\displaystyle\sum}
\newcommand{\aint}{\int_{-\infty}^{+\infty}}

% 简易图片插入
\newcommand{\qfig}[3][nolabel]{
  \begin{figure}[!htb]
      \centering
      \includegraphics[width=0.6\textwidth]{#2}
      \caption{#3}
      \label{#1}
  \end{figure}
}

% 表格
\renewcommand\arraystretch{1.5}


% 日期

\lstset{
  mathescape=false
}
\begin{document}
\fi 
\def\chapname{06hw}

% Start Here
\chapter{硬件实现}

对于硬件系统来说,一般分为同步系统和异步系统,同步系统可以做到更高的速度。

% \section{逻辑设计}

逻辑设计的技巧与规则对系统性能的影响十分明显。通过逻辑设计的语言进行硬件的描述,来实现整体的功能。


硬件和软件是相互耦合的,也因此可以进行优化,但是也出现了相应的平台依赖问题。


\section{组合逻辑电路}

组合逻辑电路都可以通过真值表实现。

% \section{时序逻辑电路}

\section{CPU 设计}

数据通路是对 ISA 的模块化实现。

过程可以分为 

\begin{itemize}
    \item 取指
    \item 译码
    \item 读存
    \item 计算
    \item 写存
\end{itemize}

\section*{术语}

MMU: Memory Management Unit



% End Here

\let\chapname\undefined
\ifx\mainclass\undefined
\end{document}
\fi 