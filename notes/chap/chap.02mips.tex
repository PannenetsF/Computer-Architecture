\ifx\mainclass\undefined
\documentclass[cn,11pt,chinese,black,simple]{../elegantbook}
% 微分号
\newcommand{\dd}[1]{\mathrm{d}#1}
\newcommand{\pp}[1]{\partial{}#1}

\newcommand{\homep}[1]{\Large\textbf{Problem #1}}
\newcommand{\subhome}[1]{\large\textbf{SubProblem #1}}

% FT LT ZT
\newcommand{\ft}[1]{\mathscr{F}[#1]}
\newcommand{\fta}{\xrightarrow{\mathscr{F}}}
\newcommand{\lt}[1]{\mathscr{L}[#1]}
\newcommand{\lta}{\xrightarrow{\mathscr{L}}}
\newcommand{\zt}[1]{\mathscr{Z}[#1]}
\newcommand{\zta}{\xrightarrow{\mathscr{Z}}}

% 积分求和号

\newcommand{\dsum}{\displaystyle\sum}
\newcommand{\aint}{\int_{-\infty}^{+\infty}}

% 简易图片插入
\newcommand{\qfig}[3][nolabel]{
  \begin{figure}[!htb]
      \centering
      \includegraphics[width=0.6\textwidth]{#2}
      \caption{#3}
      \label{#1}
  \end{figure}
}

% 表格
\renewcommand\arraystretch{1.5}


% 日期

\lstset{
  mathescape=false
}
\begin{document}
\fi 
\def\chapname{02mips}

% Start Here
\chapter{MIPS 导论:汇编指令集}

不同的核(Cluster)之间的传输通过总线,吞吐降低。改善架构存在必要。

\section{什么是汇编语言?}

汇编语言(Assembly Language)是 CPU 可以接收的基本操作,各个 CPU 系列存在不同。

\section{指令集(Instruction Set Architectures)}

随着计算机的发展,需要不同的功能,对应着生成许多的指令集不同的实现。

最初出现的 VAX 有许多的指令,可以执行很大的运算。对应的 RISC 指令集将指令变成更细粒度的实现,虽然很多的问题需要巨量的指令数目,但是速度优于 VAX ,更小的指令用量更大,带来更规整的芯片布局,从而时钟周期会更小。 RISC 阵营包括:ARM,MIPS 以及 RISC-V 。

MIPS 汇编语言贴近硬件的实现,没有变量类型的概念,操作的单元是寄存器,算数操作的来源只能是寄存器。寄存器的速度与其硬件开销存在制衡,MIPS 中只有 32 位寄存器,满足大部分的需求,并且硬件便于实现。那么这样的 32-bit 称为一个字(word)。

寄存器可以用数字或者名称引用,数字形式:\lstinline{$1, $2, ... ,$32}

定义如下:

\begin{itemize}
    \item \lstinline{$16 - $23} \(\rightarrow\) \lstinline{$s0 - $s7} 对应 C 变量
    \item \lstinline{$8 - $15} \(\rightarrow\) \lstinline{$t0 - $t7} 对应临时变量
\end{itemize}

在汇编语言中,寄存器没有类型,通过操作判断其类型。

在写 MIPS 时,需要注意添加注释(\#)。

\section{运算指令格式}

规整的格式:一个操作符加上三个操作数 \lstinline{1  2,3,4},其中

\begin{enumerate}
    \item 操作符号
    \item 目标操作数:dest
    \item 第一源操作数:src1
    \item 第二源操作数:src2
\end{enumerate}

如果需要 0 ,我们可以直接引用一个特殊的零寄存器:\lstinline{$zero$}。 MIPS 中没有原生的 \lstinline{mov} 而是使用 \lstinline{add $s0, $s1, $s2} 。同样地,可以用 \lstinline{add $zero, $zero, $s0} 用来产生流水线的气泡。

如果需要常数,我们可以使用立即数指令:\lstinline{addi $s0, $s1, 10}。

\section{内存与寄存器}

内存大而慢,寄存器小而快,有一些和内存进行交互的指令也就是数据传输指令。这类的指令要求源与目标的地址,此外还有一个偏移量 offset:\lstinline{8($t0)} 指向的是指针为 \lstinline{$t0 + 8} 的内存。


规整的格式:一个\lstinline{lw}操作符加上三个操作数 \lstinline{1  2,3(4)},其中

\begin{enumerate}
    \item 操作符号
    \item 目标寄存器位置:dest
    \item 偏移量:offset
    \item 源内存位置基址:src
\end{enumerate}


规整的格式:一个\lstinline{sw}操作符加上三个操作数 \lstinline{1  2,3(4)},其中

\begin{enumerate}
    \item 操作符号
    \item 目标内存位置:dest
    \item 偏移量:offset
    \item 源寄存器位置基址:src
\end{enumerate}


% End Here

\let\chapname\undefined
\ifx\mainclass\undefined
\end{document}
\fi 