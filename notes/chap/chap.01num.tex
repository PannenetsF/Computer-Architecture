\ifx\mainclass\undefined
\documentclass[cn,11pt,chinese,black,simple]{../elegantbook}
% 微分号
\newcommand{\dd}[1]{\mathrm{d}#1}
\newcommand{\pp}[1]{\partial{}#1}

\newcommand{\homep}[1]{\Large\textbf{Problem #1}}
\newcommand{\subhome}[1]{\large\textbf{SubProblem #1}}

% FT LT ZT
\newcommand{\ft}[1]{\mathscr{F}[#1]}
\newcommand{\fta}{\xrightarrow{\mathscr{F}}}
\newcommand{\lt}[1]{\mathscr{L}[#1]}
\newcommand{\lta}{\xrightarrow{\mathscr{L}}}
\newcommand{\zt}[1]{\mathscr{Z}[#1]}
\newcommand{\zta}{\xrightarrow{\mathscr{Z}}}

% 积分求和号

\newcommand{\dsum}{\displaystyle\sum}
\newcommand{\aint}{\int_{-\infty}^{+\infty}}

% 简易图片插入
\newcommand{\qfig}[3][nolabel]{
  \begin{figure}[!htb]
      \centering
      \includegraphics[width=0.6\textwidth]{#2}
      \caption{#3}
      \label{#1}
  \end{figure}
}

% 表格
\renewcommand\arraystretch{1.5}


% 日期

\lstset{
  mathescape=false
}
\begin{document}
\fi 
\def\chapname{01num}

% Start Here
\chapter{数据表示}

数据需要数字化之后才可以在计算机进行表示。核心的折中点:如何用尽可能少的位尽可能精确表示一个数字?

\section{表示方式}

\begin{definition}[原码]
    {s|bbb}: 数字大小的变化方向不一致,存在两个 0 : 0000 ,1111
\end{definition}

\begin{definition}[反码]
    {sssbb}: 数字大小的变化方向一致,存在两个 0 : 0000 ,1111
\end{definition}

\begin{definition}[补码]
    {sbbbb}: 数字大小的变化方向一致,存在一个 0 : 0000

    在进行位变换时,需要进行符号扩展。
\end{definition}



% \[\begin{aligned}
%     % & bb\mathbb{P a n n e n e t s . F} \\
%     & bb\mathbb{abcdefghijklmnopqrstuvwxyz} \\
%     & bb\mathbb{ABCDEFGHIGKLMNOPQRSTUVWXYZ} \\
%     & bf\mathbf{abcdefghijklmnopqrstuvwxyz} \\
%     & bf\mathbf{ABCDEFGHIGKLMNOPQRSTUVWXYZ} \\
%     & cal\mathcal{abcdefghijklmnopqrstuvwxyz} \\
%     & cal\mathcal{ABCDEFGHIGKLMNOPQRSTUVWXYZ} \\
%     & frak\mathfrak{abcdefghijklmnopqrstuvwxyz} \\
%     & frak\mathfrak{ABCDEFGHIGKLMNOPQRSTUVWXYZ} \\
%     & it\mathit{abcdefghijklmnopqrstuvwxyz} \\
%     & it\mathit{ABCDEFGHIGKLMNOPQRSTUVWXYZ} \\
%     & rm\mathrm{abcdefghijklmnopqrstuvwxyz} \\
%     & rm\mathrm{ABCDEFGHIGKLMNOPQRSTUVWXYZ} \\
%     & sf\mathsf{abcdefghijklmnopqrstuvwxyz} \\
%     & sf\mathsf{ABCDEFGHIGKLMNOPQRSTUVWXYZ} \\
%     & tt\mathtt{abcdefghijklmnopqrstuvwxyz} \\
%     & tt\mathtt{ABCDEFGHIGKLMNOPQRSTUVWXYZ} \\
%     & rm\mathscr{abcdefghijklmnopqrstuvwxyz} \\
%     & rm\mathscr{ABCDEFGHIGKLMNOPQRSTUVWXYZ} \\
%     & \mathfrak{Pannenets.F or PANNENETS.F}



% \end{aligned}\]




% End Here

\let\chapname\undefined
\ifx\mainclass\undefined
\end{document}
\fi 